\section{Порядок выполнения работы\\}

\begin{enumerate} 
\item[1.] Ознакомиться с теоретической частью лабораторной работы.
\item[2.] Получить DH параметры манипулятора с который будет отрабатывать траекторию.
\item[3.] Решить ПЗК и ОЗК.
\item[4.] Вывести ПИД коэффициенты для привода манипулятора.
\item[5.] Прогнать последнее сочленение манипулятора по синусу при помощи ПИД-регулятора и снять углы.
\item[6.] Реализовать траекторный регулятор представленный на рис. 4.4(или любой другой) и снять показание датчиков как в п.5..
(Реализовав любой другой регулятор он должен выполнять функцию траекторного регулятора)
\item[7.] Сделать моделирование в xcos.
\item[8.] Реализовать один из методов планирования траектории и нарисовать при помощи манипулятора какой-нибудь не сложный рисунок(например смайлик или домик).
 \end{enumerate}
 